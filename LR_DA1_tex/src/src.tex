\section{Описание}
Требуется написать реализацию алгоритма поразрядной сортировки. В качестве ключа выступают строки переменной длины (до 2048 символов).

Как сказано в Wikipedia: \enquote{«По аналогии с разрядами чисел будем называть элементы, из которых состоят сортируемые объекты, разрядами. Сам алгоритм состоит в последовательной сортировке объектов какой-либо устойчивой сортировкой по каждому разряду, в порядке от младшего разряда к старшему, после чего последовательности будут расположены в требуемом порядке}.

В качестве внутренней устойчивой сортировки будем использовать сортитовку подсчётом. Поразрядная сортировка в таком случае работает за $O(\frac{b}{R}(n+2^R)$, причем $O(n+2^R)$ - сложность сортировки подсчётом, где $b$ - длина числа (в нашем случае в битах, то есть $b=64$), $R$ - длина (размер) разряда, по которому мы будем сортировать, $n$ - кол-во пар \enquote{ключ-значение} в сортируемой последовательности. Так как $R$ явно не задается, вычислять его будем следующим образом:

\begin{itemize}
    \item При $b\leq\log_2 n$: $R=b$, при этом сложность выполнения программы сводится к $O(n+2^b)$, причем $n+2^b\leq2n$ 
    \item При $b>\log_2 n$: $R=\log_2n$, при этом сложность выполнения программы сводится к $O(\frac{2bn}{\log_2n})$
\end{itemize}

\pagebreak

\section{Исходный код}
Разобьем процесс написания программы на несколько этапов

\begin{enumerate}
    \item Реализация необходимых новых типов (пара \enquote{ключ-значение} и вектор)
    \item Реализация алгоритма сортировки подсчетом вектора пар по ключу по заданному разряду заданной длины
    \item Реализация алгоритма поразрядной сортировки подсчетом вектора пар
    \item Реализация ввода-вывода
\end{enumerate}

Код:
\begin{lstlisting}[language=C++]
#include <iostream>
#include <algorithm>
#include <cmath>

template <typename T>
class TVector {
public:
TVector() = default;

TVector(size_t newSize) {
capacity = newSize;
size = 0;
storage = new T[newSize];
}

TVector(const TVector& v) {
size = v.size;
capacity = v.capacity;
storage = new T[size];
for (size_t i = 0; i < size; i++) {
storage[i] = v.storage[i];
}
}

const T& operator[](size_t index) const {
return storage[index];
}

T& operator[](size_t index) {
return storage[index];
}

~TVector() {
delete[] storage;
}

size_t sizeM() const {
return size;
}

const T* begin() const {
return storage;
}

const T* end() const {
return storage + size;
}

T* Begin() {
return storage;
}

T* End() {
return storage + size;
}

void AddLast(const T& value) {
if (size < capacity) {
storage[size] = value;
++size;
return;
}
int next_size = 10000;
if (capacity > 0)
next_size = capacity * 10;

T* tempvec = new T[next_size];
std::copy(begin(), end(), tempvec);
delete[] storage;
storage = tempvec;
capacity = next_size;
storage[size] = value;
++size;
}

TVector& operator=(TVector& other) {
if (&other == this) {
return *this;
}
if (other.size <= capacity) {
std::copy(other.begin(), other.end(), begin());
size = other.size;
}
else {
delete[] storage;
storage = new T[other.capacity];
std::copy(other.begin(), other.end(), begin());
size = other.size;
capacity = other.capacity;
}
return *this;
}

TVector& operator = (TVector&& other) {
if (&other == this) {
return *this;
}
delete[] storage;
storage = other.storage;
other.storage = nullptr;
size = other.size;
other.size = 0;
capacity = other.capacity;
other.capacity = 0;
return *this;

}

private:
std::size_t capacity = 0;
std::size_t size = 0;
T* storage = nullptr;

};

struct TElem {
std::uint64_t Key;
char Value[2049];
};

struct TSortElem {
std::uint64_t Key;
std::uint64_t Ind;
};


TVector<TSortElem> Sortvec(TVector<TSortElem>vec, std::uint64_t maxKey) {
std::uint64_t h = pow(10, 19);
int i, count[10] = { 0 };
int k = vec.sizeM();
TVector<TSortElem> res(k);
for (std::uint64_t exp = 1; maxKey / exp > 0 && exp <= h; exp *= 10) {
for (i = 0; i <= 9; i++)
count[i] = 0;
for (i = 0; i < k; i++) {
count[(vec[i].Key / exp) % 10]++;
}
for (i = 1; i < 10; i++)
count[i] += count[i - 1];
for (i = k - 1; i >= 0; i--) {
res[count[(vec[i].Key / exp) % 10] - 1] = vec[i];
count[(vec[i].Key / exp) % 10]--;
}
for (i = 0; i < k; i++) {
vec[i] = res[i];
}
if (exp == h)
break;
}
return vec;
}

int main() {
std::ios_base::sync_with_stdio(false);
std::cin.tie(NULL);
TVector<TElem> vec;
TElem temp;
TVector<TSortElem> vecSort;
TVector<TSortElem> Result;
TSortElem tempSort;
int i = 0;
std::uint64_t maxKey = 0;
while (std::cin >> temp.Key >> temp.Value) {
vec.AddLast(temp);
if (temp.Key > maxKey) {
maxKey = temp.Key;
}
tempSort.Key = vec[i].Key;
tempSort.Ind = i;
vecSort.AddLast(tempSort);
i++;
}

Result = Sortvec(vecSort, maxKey);

int M = vec.sizeM();

for (i = 0; i < M; i++) {
tempSort = Result[i];
std::cout << tempSort.Key << " " << vec[tempSort.Ind].Value << "\n";

}
}
\end{lstlisting}

\begin{longtable}{|p{7.5cm}|p{7.5cm}|}
\hline
\rowcolor{lightgray}
\multicolumn{2}{|c|} {vector.h}\\\hline
TVector()&Конструктор по умолчанию\\\hline
explicit TVector(size\_t newSize,&Конструктор от двух аргументов:\\ 
const T\& defaultValue = T()) : TVector()&размер вектора и значения по умолчанию\\\hline
TVector(const TVector\& other)&Конструктор копирования\\ 
: TVector();&\\\hline
TVector\& operator=(TVector other)&Оператор присваивания с копированием\\\hline
~TVector()&Деструктор\\\hline
size\_t Size() const&Размер вектора\\\hline
bool Empty() const&Проверка на пустоту\\\hline
T* begin() const&Итератор на начало\\\hline
T* end() const&Итератор за конец\\\hline
static void Swap(TVector\& lhs, TVector\& rhs)&Обмен векторов значениями\\\hline
void PushBack(const T\& value)&Добавление элемента в конец\\\hline
const T\& At(size\_t index) const&Получение константной ссылки на элемент по индексу\\\hline
T\& At(size\_t index)&Получение ссылки на элемент по индексу\\\hline
const T\& operator[](size\_t index) const&Получение константной ссылки на элемент по индексу\\\hline
T\& operator[](size\_t index)&Получение ссылки на элемент по индексу\\\hline
\end{longtable}

\pagebreak

\section{Консоль}

\begin{alltt}
(py37) ~ cd da1                    
(py37) ~ g++ -pedantic -Wall -Wextra main.cpp -o out      
(py37) ~ da1 ./out <test.txt >res.txt  
(py37) ~ da1 cat test.txt              
0 xGfxrxGGxrxMMMMfrrrG
18446744073709551615  xGfxrxGGxrxMMMMfrrr
0 xGfxrxGGxrxMMMMfrr
18446744073709551615  xGfxrxGGxrxMMMMfr
(py37) ~ da1 cat res.txt               
0 xGfxrxGGxrxMMMMfrrrG
0 xGfxrxGGxrxMMMMfrr
18446744073709551615 xGfxrxGGxrxMMMMfrrr
18446744073709551615 xGfxrxGGxrxMMMMfr
\end{alltt}

\pagebreak

