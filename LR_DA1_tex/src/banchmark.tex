\section{Тест производительности}

Тест производительности представляет из себя следующее: сортировку 10 миллионах входных данных с помощью реализованной поразрядной сортировки и $std::stable\_sort$.\\

Моя реализация:
\begin{alltt}
windicor@Windicor:~/OOP_lab1_console\$ ./a.out <test4.txt >res.txt
input3294 ms
sort2191 ms
output2106 ms
\end{alltt}

$std::stable\_sort$:
\begin{alltt}
windicor@Windicor:~/OOP_lab1_console\$ ./a.out <test4.txt >res.txt
input3301 ms
sort2242 ms
output2078 ms
\end{alltt}

Как видно, поразрядная сортировка работает несколько быстрее, но, всё же, недостаточно быстро, чтобы продемонстрировать разницу в ассимптотике между $O(n)$ и $O(\log_2n)$. Предположу, что происходит это в силу универсальности алгоритма (использование шаблонных функций) и создания на каждой итерации сортировки при вызове $CountingSort$ промежуточного вектора размера $n$.

\pagebreak

