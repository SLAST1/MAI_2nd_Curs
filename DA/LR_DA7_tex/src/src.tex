\section{Описание}
Как описано в [1], динамическое программирование - это метод решения задач, при
котором сложная задача разбивается на более простые, решение сложной задачи
составляется из решений простых задач.

Этот метод очень похож на «разделяй и властвуй», но динамическое программирование допускает использование метода восходящего анализа, который позволяет
изначально решать простые задачи и получать на их базе решение более сложных.

Так же метод запоминает решения подзадач, потому что часто для построения нужно
обращаться за оптимальным решением к одним и тем же малым задачам.

Задача о рюкзаке является известной NP-полной задачей, которая при некоторых
ограничениях решается за полиномиальное время с помощью метода динамического
программирования.

Стадартный вариант задачи описан и доказан в [2]. Для моего варианта задания
$dp_i,j,k$ - максимальная стоимость $j$ вещей из первых $i$, таких, что их суммарный вес
не превышает $k$. То есть алгоритм будет перебирать количество предметов, которые будут в рюкзаке.

Пусть существует оптимальное решение в $dp_i,j,k-w_j-1$, тогда $dp_i+1,j+1,k = max(dp_i,j,k-wj-1 + c_j+1, dpi + 1, j, k)$. В рекуррентной формуле рассматривается два варианта: взять вещь $j + 1$ или нет.

Такое решение имеет $n^2 \times m$ состояния, в каждое можно перейти из двух других. Так временная сложность алгоритма $O(n^2 \times m)$.

Хранение всей таблицы состояний слишком дорого по памяти, но необходимо для восстановления ответа. 
Поэтому будем хранить только $dp_i$ и $dp_i+1$ и битовые множества предметов, которые оптимальны для решения подзадачи.
Пространственная сложность такого подхода $O(n \times m)$.
\pagebreak

\section{Исходный код}
Опишем матрицы $dpPrev$ и $dpCur$ для $dp_j+1$ и $dp_j$, матрицы $setCur$ и $setPrev$ для хранения множества предметов.
Для достижения пространственной сложности $O(n \times m)$ будем использовать эффективный по памяти $std :: bitset$.

Код: main.cpp
\begin{lstlisting}[language=C++]
#include <bitset>
#include <iostream>
#include <vector>

const size_t MAX_N = 100;

int main() {
int n, m;
std::cin >> n >> m;
std::vector<int> w(n);
std::vector<long long> c(n);
for (int i = 0; i < n; ++i) {
std::cin >> w[i] >> c[i];
}

std::vector< std::vector< long long > > dpPrev(n + 1, std::vector<long long>(m + 1));
std::vector< std::vector< long long > > dpCur(n + 1, std::vector<long long>(m + 1));
std::vector< std::vector< std::bitset<MAX_N> > > setPrev(n + 1, std::vector< std::bitset<MAX_N> >(m + 1));
std::vector< std::vector< std::bitset<MAX_N> > > setCur(n + 1, std::vector< std::bitset<MAX_N> >(m + 1));
long long ans = 0;
std::bitset<MAX_N> res;

for (int j = 1; j < n + 1; ++j) {
for (int k = 1; k < m + 1; ++k) {
dpPrev[j][k] = dpPrev[j - 1][k];
setPrev[j][k] = setPrev[j - 1][k];
if (c[j - 1] > dpPrev[j][k] and k - w[j - 1] == 0) {
dpPrev[j][k] = c[j - 1];
setPrev[j][k] = 0;
setPrev[j][k][j - 1] = 1;
}
if (dpPrev[j][k] > ans) {
ans = dpPrev[j][k];
res = setPrev[j][k];
}
}
}

for (long long i = 2; i < n + 1; ++i) {
for (int j = 1; j < n + 1; ++j) {
for (int k = 1; k < m + 1; ++k) {
dpCur[j][k] = dpCur[j - 1][k];
setCur[j][k] = setCur[j - 1][k];
if (k - w[j - 1] > 0 and dpPrev[j - 1][k - w[j - 1]] > 0) {
if (i * (c[j - 1] + dpPrev[j - 1][k - w[j - 1]] / (i - 1)) > dpCur[j][k]) {
dpCur[j][k] = i * (c[j - 1] + dpPrev[j - 1][k - w[j - 1]] / (i - 1));
setCur[j][k] = setPrev[j - 1][k - w[j - 1]];
setCur[j][k][j - 1] = 1;
}
}
if (dpCur[j][k] > ans) {
ans = dpCur[j][k];
res = setCur[j][k];
}
}
}
std::swap(dpCur, dpPrev);
std::swap(setCur, setPrev);
}
std::cout << ans << '\n';
for (int i = 0; i < n; ++i) {
if (res[i]) {
std::cout << i + 1 << ' ';
}
}
std::cout << '\n';
}

\end{lstlisting}

\pagebreak

\section{Консоль}

\begin{alltt}
(py37) ~ /DA_labs/lab7$ make
g++ -g -O2 -pedantic -std=c++17 -Wall -Wextra -Werror main.cpp -o solution
(py37) ~ /DA_labs/lab7$ cat tests/1.in
3 6
2 1
5 4
4 2
(py37) ~ /DA_labs/lab7$ ./solution <tests/1.in
6
1 3
(py37) ~ /DA_labs/lab7$ cat tests/2.in
14 41
2 60
6 25
10 56
8 4
7 81
4 40
10 56
7 2
8 32
2 25
6 22
9 5
9 95
8 6
(py37) ~ /DA_labs/lab7$ ./solution <tests/2.in
2674
1 2 3 5 6 10 13
\end{alltt}

\pagebreak

