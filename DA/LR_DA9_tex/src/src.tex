\section{Описание}
Для написания алгоритма Джонсона нам понадобится написать еще два алгоритма, алгоритм Беллмана-Форда и Дейкстры. Сам алгоритм таков, если алгоритм
Беллмана-Форда возвращает ложь, то у нас отрицательный цикл, иначе используем
алгоритм Дейкстры, если нет отрицательных дуг, иначе формируем новый орграф
с такими же кратчайшими путями, но без отрицательных дуг и снова запускаем
алгоритм Дейкстры.

\pagebreak

\section{Исходный код}

Код: main.cpp
\begin{lstlisting}[language=C++]
#include "johnson.hpp"
#include "structures.hpp"

int main()
{
size_t n = 0;
size_t m = 0;
std::cin >> n >> m;
TGraph gr(n, m);
TMatrix dist(n, std::vector<int64_t>(n, INT64_MAX));
for(size_t i = 0; i < n; ++i)
dist[i][i] = 0;
size_t from = 0;
size_t to = 0;
int64_t weight = 0;
for(size_t k = 0; k < m; ++k)
{
std::cin >> from >> to >> weight;
gr.edges.push_back(TEdge{from - 1, to - 1, weight});
}
if(Johnson(gr, dist))
{
for(size_t i = 0; i < n; ++i)
{
for(size_t j = 0; j < n; ++j)
{
if(dist[i][j] == INT64_MAX)
std::cout << "inf";
else
std::cout << dist[i][j];
if(j != n - 1)
std::cout << ' ';
}
std::cout << "\n";
}
}
return 0;
}
\end{lstlisting}
\pagebreak

Код: johnson.cpp
\begin{lstlisting}[language=C++]
#include "johnson.hpp"
#include "structures.hpp"

bool operator<(TEdge const& p1, TEdge const& p2) { return p1.weigth > p2.weigth; }

void Deikstra(TMatrix const& gr, size_t const& node, TMatrix& dist, size_t const& n)
{
dist[node][node] = 0;
std::priority_queue<TEdge> pq;
TEdge t = {node, 0, 0};
pq.push(t);
while(!pq.empty())
{
TEdge s = pq.top();
pq.pop();
for(size_t i = 0; i < n; ++i)
{
if(dist[node][i] - dist[node][s.from] > gr[s.from][i])
{
dist[node][i] = dist[node][s.from] + gr[s.from][i];
TEdge p = { i, 0, dist[node][i] };
pq.push(p);
}
}
}
}

bool BellmanFord(TGraph const& gr, size_t const& node, TMatrix& dist)
{
dist[node][node] = 0;
for(size_t j = 0; j < gr.v - 1; ++j)
{
for(auto& i: gr.edges)
if(dist[node][i.from] != INT64_MAX &&  dist[node][i.to] > dist[node][i.from] + i.weigth)
dist[node][i.to] = (dist[node][i.from] + i.weigth);
}
for(auto& i: gr.edges)
if(dist[node][i.from] != INT64_MAX &&  dist[node][i.to] > dist[node][i.from] + i.weigth)
return false;
return true;
}

bool Johnson(TGraph const& gr, TMatrix& dist)
{
TGraph new_gr;
new_gr.v = gr.v + 1;
new_gr.e = gr.e + gr.v;
new_gr.edges = gr.edges;
for(size_t i = 0; i < gr.v; ++i)
new_gr.edges.push_back(TEdge{gr.v, i, 0});
TMatrix new_dist(1, std::vector<int64_t>(new_gr.v, 0));
if(!BellmanFord(new_gr, 0, new_dist))
{
std::cout << "Negative cycle\n";
return false;
}
TMatrix graph(gr.v, std::vector<int64_t>(gr.v, INT64_MAX));
for(size_t i = 0; i < gr.v; ++i)
graph[i][i] = 0;
for(auto& i: gr.edges)
graph[i.from][i.to] = i.weigth + new_dist[0][i.from] - new_dist[0][i.to];
for(size_t i = 0; i < gr.v; ++i)
Deikstra(graph, i, dist, gr.v);
for(size_t i = 0; i < gr.v; ++i)
for(size_t j = 0; j < gr.v; ++j)
if(dist[i][j] != INT64_MAX)
dist[i][j] = dist[i][j] + new_dist[0][j] - new_dist[0][i];
return true;
}

\end{lstlisting}
\pagebreak

\section{Консоль}

\begin{alltt}
(py37) ~ /DA_labs/lab8$ make
g++ -g -O2 -pedantic -std=c++17 -Wall -Wextra -Werror main.cpp -o solution
(py37) ~ /DA_labs/lab8$ cat test.txt
5 4
1 2 -1
2 3 2
1 4 -5
3 1 1
(py37) ~ /DA_labs/lab8$ ./solution <test.txt
0 -1 1 -5 inf
3 0 2 -2 inf
1 0 0 -4 inf
inf inf inf 0 inf
inf inf inf inf 0
\end{alltt}

\pagebreak

