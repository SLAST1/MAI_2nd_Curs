\section{Выводы}

Пожалуй, основной вывод, который я сделал за время отладки программы для ЛР2
заключается в том, что вместо того чтобы две недели пытаться найти ошибку различными альтернативными путями лучше сразу написать тестер, ибо рано или поздно
делть его все равно придется. К тому же, я заметил у себя склонность к поиску
сложных и неочевидных ошибок, при том что, как нетрудно заметить, моя программа содержит исключительно до предела простые ошибки, и то что я за время отладки
насквозь пронизал ее путями выброса исключений о каждом несработавшем внутри
new malloc-е никак в итоге не помогло. В ходе выполнения этой лабораторной работы я также поближе познакомился с программой valgrind и научился более ли менее
успешно отыскивать утечки памяти. Так как за время поиска олибок выполнения,
я заодно натолько, насколько мог оптимизировал работу своей программы, никаких
ощутимых проблем связанных со скоростью ее работы я не испытал.
\pagebreak
