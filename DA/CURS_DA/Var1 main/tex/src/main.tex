\section{Эвристический поиск в графе}
Алгоритм поиска по первому наилучшему совпадению на графе, то есть A* просматривает все пути пошагово, ведущие от начальной вершины в конечную, пока не найдёт минимальный. Он просматривает сначала те маршруты, которые «кажутся» ведущими к цели. От жадного алгоритма, который тоже является алгоритмом поиска по первому лучшему совпадению, его отличает то, что при выборе вершины он учитывает, помимо прочего, весь пройденный до неё путь. Составляющая $g(x)$ - это стоимость пути от начальной вершины, а не от предыдущей, как в жадном алгоритме.
В начале работы просматриваются узлы, смежные с начальным; выбирается тот из них, который имеет минимальное значение $f(x)$, после чего этот узел раскрывается. На каждом этапе алгоритм оперирует с множеством путей из начальной точки до всех ещё не раскрытых (листовых) вершин графа — множеством частных решений, которое размещается в очереди с приоритетом. Приоритет пути определяется по значению $f(x) = g(x) + h(x)$. Алгоритм продолжает свою работу до тех пор, пока значение $f(x)$ целевой вершины не окажется меньшим, чем любое значение в очереди, либо пока всё дерево не будет просмотрено. Из множества решений выбирается решение с наименьшей стоимостью.

Алгоритм A* и допустим, и обходит при этом минимальное количество вершин, благодаря тому, что он работает с «оптимистичной» оценкой пути через вершину. Оптимистичной в том смысле, что, если он пойдёт через эту вершину, у алгоритма «есть шанс», что реальная стоимость результата будет равна этой оценке, но никак не меньше. Но, поскольку A* является информированным алгоритмом, то есть таким алгоритмом, который использует знания, относящиеся к конкретной задаче, такое равенство может быть вполне возможным.

Когда A* завершает поиск, он нашёл путь, истинная стоимость которого меньше, чем оценка стоимости любого пути через любой открытый узел. Но поскольку эти оценки являются оптимистичными, соответствующие узлы можно без сомнений отбросить. Таким образом, A* никогда не упустит возможности минимизировать длину пути, и потому является допустимым.

Предположим теперь, что некий алгоритм B вернул в качестве результата путь, длина которого больше оценки стоимости пути через некоторую вершину. На основании эвристической информации, для алгоритма B нельзя исключить возможность, что этот путь имел и меньшую реальную длину, чем результат. Соответственно, пока алгоритм B просмотрел меньше вершин, чем A*, он не будет допустимым. 
	
\pagebreak

\section{Код}

\begin{lstlisting}[language=C]
#include <iostream>
#include <vector>
#include <cmath>
#include <queue>
#include <iomanip>

using namespace std;

struct coords {
double x;
double y;
};

double dist(const coords &f1, const coords &f2) {
return sqrt((f1.x - f2.x) * (f1.x - f2.x) + (f1.y - f2.y) * (f1.y - f2.y));
}

class approx {
public:
int index;
double path;
double distance;
friend bool operator< (const approx &f1, const approx &f2) {
if ((f1.path + f1.distance) != (f2.path + f2.distance))
return (f1.path + f1.distance) > (f2.path + f2.distance);
return f1.index < f2.index;
}
approx(const int &new_index, const coords &u, const coords &v, const double &new_path) {
index = new_index;
path = new_path;
distance = dist(u, v);
}
};

double res(const int u, const int v, int n, const vector<coords> &vert, const vector<vector<int>> &g) {
vector<double> d(n, -10000);
priority_queue<approx> pq;
d[u] = 0;
pq.push(approx(u, vert[u], vert[v], 0));

while (!pq.empty()) {
approx tmp = pq.top();
pq.pop();
if (tmp.index == v)
break;
if (tmp.path > d[tmp.index])
continue;

for (int i = 0; i < g[tmp.index].size(); ++i) {
int next_vert = g[tmp.index][i];
if (d[next_vert] < 0 || (d[tmp.index] + dist(vert[tmp.index], vert[next_vert])) < d[next_vert]){
d[next_vert] = d[tmp.index] + dist(vert[tmp.index], vert[next_vert]);
pq.push(approx(next_vert, vert[next_vert], vert[v], d[next_vert]));
}
}
}
return d[v];
}

int main()
{
int n, m;
cin >> n >> m;
vector<coords> vert(n);
for (auto &i : vert)
cin >> i.x >> i.y;
vector<vector<int>> g(n);

for (int i = 0; i < m; ++i) {
int u, v;
cin >> u >> v;
g[u - 1].push_back(v - 1);
g[v - 1].push_back(u - 1);
}
int q;
cin >> q;

while (q) {
int u, v;
cin >> u >> v;
double solve = res(u - 1, v - 1, n, vert, g);
if (solve > 0)
cout << fixed << setprecision(6) << solve << '\n';
else
cout << "-1\n";
--q;
}
return 0;
}
\end{lstlisting}

\pagebreak