\CWHeader{Лабораторная работа \textnumero 6}

\CWProblem{
Необходимо разработать программную библиотеку на языке С или С++, реализующую простейшие арифметические действия и проверку условий над целыми
неотрицательными числами. На основании этой библиотеки нужно составить программу, выполняющую вычисления над парами десятичных чисел и выводящую результат на стандартный файл вывода.
Разработать программу на языке C или C++, реализующую построенный алгоритм.

Список арифметических операций:

Сложение.

Вычитание.

Умножение.

Возведение в степень.

Деление.

В случае возникновения переполнения в результате вычислений, попытки вычесть из
меньшего числа большее, деления на ноль или возведении нуля в нулевую степень,
программа должна вывести на экран строку Error.

Список условий:
Больше.

Меньше.

Равно.

В случае выполнения условия программа должна вывести на экран строку true, в противном случае — false.
Количество десятичных разрядов целых чисел не превышает 100000. Основание выбранной системы счисления для внутреннего представления «длинных» чисел должно быть не меньше 10000.

{\bfseries Формат входных данных:} Входной файл состоит из последовательности заданий, каждое задание состоит из трех строк:

Первый операнд операции.
Второй операнд операции.

Символ арифметической операции или проверки условия.

Числа, поступающие на вход программе, могут иметь «ведущие» нули.

{\bfseries Формат результата:} Для каждого задания из выходного файла нужно распечатать результат на отдельной строке в выходном файле:

Числовой результат для арифметических операций.

Строку Error в случае возникновения ошибки при выполнении арифметической операции. Строку true или false при выполнении проверки условия.

В выходных данных вывод чисел должен быть нормализован, то есть не содержать в себе «ведущих» нулей.

}
\pagebreak

