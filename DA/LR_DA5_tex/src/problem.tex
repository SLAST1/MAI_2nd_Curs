\CWHeader{Лабораторная работа \textnumero 5}

\CWProblem{
Необходимо реализовать алгоритм Укконена построения суффиксного дерева за линейное время. Построив такое дерево для некоторых из выходных строк,
необходимо воспользоваться полученным суффисным деревом для решения задания.

Алфавит строк: строчные буквы латинского алфавита (т.е. от a до z).

Формат входных данных:
Текст располагается на первой строке, затем, до конца файла, следуют строки с образцами.

Формат выходных данных:
Для каждого образца, найденного в тексте, нужно распечатать строку, начинающуюся с последовательного номера этого образца и двоеточия,
за которым, через запятую, нужно перечислить номера позиций, где встречается
образец в порядке возрастания.

{\bfseries Вариант:} Найти в заранее известном тексте поступающие на вход образцы.

}
\pagebreak

