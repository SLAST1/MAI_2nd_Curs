\section{Выводы}
Выполнив курсовой проект я изучил эвристический поиск в графе. Порядок обхода вершин определяется эвристической функцией. Эта функция - сумма двух других: функции стоимости достижения рассматриваемой вершины $(x)$ из начальной (обычно обозначается как $g(x)$ и может быть как эвристической, так и нет), и функции эвристической оценки расстояния от рассматриваемой вершины к конечной (обозначается как $h(x)$).
Функция $h(x)$ должна быть допустимой эвристической оценкой, то есть не должна переоценивать расстояния к целевой вершине. Например, для задачи маршрутизации $h(x)$ может представлять собой расстояние до цели по прямой линии, так как это физически наименьшее возможное расстояние между двумя точками. Итак, A* проходит наименьшее количество вершин графа среди допустимых алгоритмов, использующих такую же точную (или менее точную) эвристику.
\pagebreak